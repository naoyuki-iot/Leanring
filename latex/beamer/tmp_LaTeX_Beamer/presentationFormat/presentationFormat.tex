\documentclass[10pt]{beamer}
%%%
%% platex start.tex
%% dvips start.dvi
%% ps2pdf start.ps
%%

\mode<presentation>
{
  \usetheme{Warsaw}
  % or ...

  \setbeamercovered{transparent}
  % or whatever (possibly just delete it)
}
\usefonttheme{structurebold}


\usepackage{latexsym}
\usepackage{graphicx}
\usepackage{color}
\usepackage{multirow}
\usepackage{multimedia}
\usepackage{ascmac}


\title{タイトル}
\subtitle{サブタイトル}

\author{第1著者\inst{1} \and 第2著者\inst{2}}
\institute[The Univ. of Shiga Pref.] % (optional, but mostly needed)
{
  \inst{1}
  第1著者所属 \\
  第1著者所属部局
  \and
  \inst{2}
  第2著者所属 \\
  第2著者所属部局
}

\date{2016年5月17日}

%%
%ロゴマークを入れる場合
\logo{\includegraphics[height=0.05\textheight]{logo-usp.eps}}

\begin{document}

\begin{frame}
  \titlepage
\end{frame}


\begin{frame}{背景}
  
\end{frame}


\begin{frame}{背景}
  
  \begin{block}{ブロックタイトル}
    ブロックの内容
  \end{block}
\end{frame}



\begin{frame}{columns環境}
  \begin{columns}
    \column{0.5\textwidth}
    1列目
    %
    \column{0.5\textwidth}
    2列目
  \end{columns}
    
\end{frame}



\begin{frame}{目標}
  \begin{itemize}
  \item<1-> 評価スコアと音声信号の特徴量との関係

    \begin{columns}[t]
      % 
      \begin{column}{0.4\textwidth}
        \begin{itembox}[l]{嚥下評価スコア}
          \begin{itemize}
          \item 喉頭蓋谷や梨状陥凹などの唾液貯留
          \item 咳反射・声門閉鎖反射
          \item 嚥下反射の惹起
          \item 着色水嚥下咽頭クリアランス
          \end{itemize}
        \end{itembox}
      \end{column}
      % 
      \begin{column}{0.05\textwidth}
        \mbox{}\\[3zw]
        $\Longleftrightarrow$ 
      \end{column}
      %
      \begin{column}{0.4\textwidth}
        \begin{itembox}[l]{音声信号の特徴量}
          \begin{itemize}
          \item 時間領域
          \item 周波数領域
          \item 時間周波数領域
          \end{itemize}
        \end{itembox}
      \end{column}
      % 
    \end{columns}
    
    
  \item<2-> 健常者を対象とした深度画像による喉頭運動計測

    甲状軟骨の3次元的な軌跡を求める
  \end{itemize}
\end{frame}



\begin{frame}{現状: 音声信号の取得と解析}

  \begin{columns}[t]
    %
    \begin{column}{0.4\textwidth}
      \centering
      % \includegraphics[width=\textwidth]{figs/timePlotHealthyPerson.eps} \\
      \parbox{0.9\textwidth}{\small 健常者の音声信号の時間プロット}
    \end{column}
    %
    \begin{column}{0.4\textwidth}
      % \includegraphics[width=\textwidth]{figs/freqPlotHealthyPerson.eps} \\
      {\small 健常者の音声信号の振幅スペクトル}
    \end{column}
  \end{columns}

  \begin{center}
    \parbox{0.8\textwidth}
    {\small Mann-Whiteny U検定(多重検定補正)を用いてそれぞれの特徴量が各群間で
      有意差があるかどうかを検定した結果}

    \begin{tabular}{c*{4}{p{0.15\textwidth}}} \hline
      比較対象群 & {\scriptsize 振幅スペクトルの最大値}
      & {\scriptsize ピーク周波数}
      & {\scriptsize 周波数成分の加重平均}
      & {\scriptsize 周波数成分の分散} \\ \hline
      健常群 -- 軽度群 & あり & あり & なし & あり \\
      健常群 -- 重度群 & あり & あり & なし & あり \\
      軽度群 -- 重度群 & なし & あり & なし & あり \\ \hline
    \end{tabular}
  \end{center}
\end{frame}



\begin{frame}{現状: 甲状軟骨の前後運動}
マーカ位置の喉頭前後運動の観測例
\begin{columns}[b]
  %
  \begin{column}{0.3\textwidth}
    %\includegraphics[width=\textwidth]{figs/colorMarker.eps}

    %\includegraphics[width=\textwidth]{figs/colorMarkerDetection.eps} 
  \end{column}
  %
  \begin{column}{0.6\textwidth}
    %\includegraphics[width=\textwidth]{figs/sampleDepthVariation.eps}
  \end{column}
\end{columns}


\end{frame}



\end{document}
